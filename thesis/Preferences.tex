%%%%%%%%%%%%%%%%%%%%%%%%%%%%%%%%%%%%%%%%%%%%%%%%%%%%%%%%%%%
% Allgemeines
\usepackage[automark]{scrlayer-scrpage}	% Kopf- und Fusszeilen
\usepackage{hyperref} 					% Internetseiten
\usepackage{acro}		% Abkuerzungen					
%\usepackage{showframe}
\usepackage{abstract}

\acsetup{list/display=all}

%%%%%%%%%%%%%%%%%%%%%%%%%%%%%%%%%%%%%%%%%%%%%%%%%%%%%%%%%%%
% Encoding 
\usepackage[T1]{fontenc}
\usepackage{lmodern}
\usepackage[utf8]{inputenc} 							% UTF8-Kodierung fuer Umlaute usw
\usepackage[ngerman, english]{babel}								% Deutsche Rechtschreibung
%\RequirePackage[ngerman=ngerman-x-latest]{hyphsubst}	% Silbentrennung
\usepackage{csquotes}
\usepackage{microtype,textcomp}		


%%%%%%%%%%%%%%%%%%%%%%%%%%%%%%%%%%%%%%%%%%%%%%%%%%%%%%%%%%%
% Bildunterschrift
\usepackage[font=small,  justification=centering,singlelinecheck=false,
			labelfont=bf]{caption}		% Formatierung von captions
			
\setcapindent{0em} 						% kein Einruecken der Caption von Figures und Tabellen
\setcapwidth{0.9\textwidth}
\setlength{\abovecaptionskip}{0.2cm} 	% Abstand der zwischen Bild- und Bildunterschrift
%\captionsetup[table]{justification=justified,singlelinecheck=false,margin={-0cm,0cm}}
% \captionsetup[subfigure]{oneside,margin={-0cm,0cm}}
% \captionsetup[algorithm]{labelsep=colon,justification=justified,singlelinecheck=false,margin={-0cm,0cm}}

\usepackage{float}
\floatstyle{plaintop}
\restylefloat{algorithm}


%%%%%%%%%%%%%%%%%%%%%%%%%%%%%%%%%%%%%%%%%%%%%%%%%%%%%%%%%%%
% Mathematik
\usepackage{mathtools} % Mathematik
\usepackage{amsfonts} 
\usepackage{amssymb}
\usepackage{siunitx}
%sudo apt -y install texlive-science

% Mathematik-Operatoren
\DeclareMathOperator{\atantwo}{atan2}
\DeclareMathOperator*{\argmin}{\arg\!\min}

%%%%%%%%%%%%%%%%%%%%%%%%%%%%%%%%%%%%%%%%%%%%%%%%%%%%%%%%%%%
% Tabellen
%\usepackage{multirow} 	% Tabellen-Zellen ueber mehrere Zeilen
\usepackage{tabularray}
% \usepackage{multicol} 	% mehre Spalten auf eine Seite
% \usepackage{tabularx} 	% Fuer Tabellen mit vorgegeben Groessen
% \usepackage{longtable}	% Ausrichtung des Abkürzungsverzeichnisses
% \usepackage{rotating}	% Gedrehte Tabellen (sidewaystable)
% \usepackage{colortbl} 	% Farbe für Zelle
% \usepackage{ragged2e}	% Rechtsbündige P Spalte
%\usepackage{tablefootnote}
\UseTblrLibrary{booktabs}


%%%%%%%%%%%%%%%%%%%%%%%%%%%%%%%%%%%%%%%%%%%%%%%%%%%%%%%%%%%
% Bilder
\usepackage{graphicx} 				% Bilder
\usepackage{epstopdf} 				% enable eps graphics
\usepackage{subcaption} 				% mehrere Abbildungen nebeneinander/uebereinander
\usepackage[dvipsnames]{xcolor} % Farben
%\usepackage[export]{adjustbox}		% Rahmen fuer Bilder (cframe=lightgray)
\usepackage{pgf}
\usepackage{svg}


%%%%%%%%%%%%%%%%%%%%%%%%%%%%%%%%%%%%%%%%%%%%%%%%%%%%%%%%%%%
% Quellcode
\usepackage{listings} 					% fuer Formatierung in Quelltexten
\usepackage{scrhack}  					% removes Warning: \float@addtolists detected!  Caused by package listings
% \usepackage[final,outer]{struktex} 	  	% Struktugramme
% \usepackage[plain]{algorithm}
% \usepackage[noend]{algpseudocode} 		% pseudo code


%%%%%%%%%%%%%%%%%%%%%%%%%%%%%%%%%%%%%%%%%%%%%%%%%%%%%%%%%%%
% Zeichnungen/Plots
\usepackage{pgfplots}						% Plots
\usetikzlibrary{matrix, positioning, arrows.meta, chains, shapes}						% Matrizen zeichnen
\usepackage{tikz}							% tikz picture
\usepackage{tikz-cd}
\usepackage{tikz-qtree,tikz-qtree-compat}	% Bäume
\usepackage{dirtree}						% Dateistrukutr
\pgfplotsset{compat=1.15}

%%%%%%%%%%%%%%%%%%%%%%%%%%%%%%%%%%%%%%%%%%%%%%%%%%%%%%%%%%%
% Sonstiges:
% Durchgehende Nummerierung. Ohne Chapter
\usepackage{chngcntr}
\counterwithout{equation}{chapter}
\counterwithout{figure}{chapter}
\counterwithout{table}{chapter}





%%%%%%%%%%%%%%%%%%%%%%%%%%%%%%%%%%%%%%%%%%%%%%%%%%%%%%%%%%%
%
% Einstellungen zum Literaturverzeichnis
%
% Hier eine von zwei Varianten auswaehlen: Nummern oder Buchstaben fuer Referenzen
%
%\usepackage[backend=biber, style=alphabetic, sorting=nyt]{biblatex}
%sudo apt-get install texlive-bibtex-extra biber
\usepackage[
	backend=biber, 
	style=numeric-comp, 
	sorting=none,
	backref=true,
	backrefstyle=three+
	]{biblatex}
	
	
\ExecuteBibliographyOptions{%
     maxbibnames=99,   % Alle Autoren (kein et al.)
     maxcitenames=1,   % Kuerzel nur aus 1. Autor im Text
     maxalphanames=1,  % nur 1. Autor in der Abkuerzung
     backref=false,    % keine Ruueckverweise auf Zitatseiten
     giveninits=true,  % Vornamen abkuerzen
     isbn=false,       % ISBN ausblenden
     doi=false,        % DOI ausblenden
   }
   
   
\DefineBibliographyStrings{ngerman}{
   andothers = {{et\,al\adddot}},            
} 
   
   
\renewcommand*{\labelalphaothers}{} % alpha label ohne +
%
\renewbibmacro*{volume+number+eid}{%
     \setunit{\space}\printfield{volume}%
     \iffieldundef{number}{}{%
      \printtext[parens]{\printfield{number}}}%
     \setunit{\addcomma\space}\printfield{eid}}
%
% no word 'pages' for articles in the bibliography (print as is)
\DeclareFieldFormat[article, inproceedings, incollection, unpublished]{pages}{#1} 
% no quotes for article titles (print as is)
\DeclareFieldFormat[article, inproceedings, incollection, online, unpublished]{title}{#1} 
%
\renewbibmacro*{date}{\printdate}
\renewbibmacro*{issue+date}{\usebibmacro{issue}}
\renewbibmacro*{publisher+location+date}{\printlist{publisher}}
%
   \setcounter{biburlnumpenalty}{9000}
   \setcounter{biburlucpenalty}{9000}
   \setcounter{biburllcpenalty}{9999}
%
% "In:" removed for articles; issue/date macros added after note+pages macro
\DeclareBibliographyDriver{article}{%
  \usebibmacro{bibindex}%
  \usebibmacro{begentry}%
  \usebibmacro{author/translator+others}%
  \setunit{\labelnamepunct}\newblock%
  \usebibmacro{title}%
  \newunit%
  \printlist{language}%
  \newunit\newblock%
  \usebibmacro{byauthor}%
  \newunit\newblock%
  \usebibmacro{bytranslator+others}%
  \newunit\newblock%
  \printfield{version}%
  \newunit\newblock%
  \usebibmacro{journal+issuetitle}%
  \newunit%
  \usebibmacro{byeditor+others}%
  \newunit%
  \usebibmacro{note+pages}%
  \setunit{\addcomma\addspace}%
  \usebibmacro{date}%
  \usebibmacro{finentry}}
%
%
\DeclareBibliographyDriver{inproceedings}{%
    \usebibmacro{begentry}%
    \printnames{author}%
    \setunit{\labelnamepunct}\newblock%
    \printfield{title}%
    \setunit{\labelnamepunct}%
	\usebibmacro{in:}%    
    \newblock%
    \ifnameundef{editor}%
    {%
    		\setunit{\adddot\space}%
    		\newunit%
    }%
    {%
     	\setunit{\addspace}%
     	\printnames[byeditor]{editor}%
     	\clearname{editor}%
     	\setunit{\space}%
     	\printtext[parens]{Hrsg.}%
     	\setunit{\addcolon\space}%
     	\newunit%
     }%
	\printfield{booktitle}%
	\setunit{\addcomma\space}%
	\printfield{pages}%
	\setunit{\addcomma\space}%
    \usebibmacro{date}%
    \usebibmacro{finentry} 
}

\DeclareBibliographyDriver{book}{%
  \usebibmacro{bibindex}%
  \usebibmacro{begentry}%
  \usebibmacro{author/editor+others/translator+others}%
  \setunit{\labelnamepunct}\newblock
  \usebibmacro{maintitle+title}%
  \newunit
  \printlist{language}%
  \newunit\newblock
  \usebibmacro{byauthor}%
  \newunit\newblock
  \usebibmacro{byeditor+others}%
  \newunit\newblock
  \printfield{edition}%
  \newunit
  \iffieldundef{maintitle}
    {\printfield{volume}%
     \printfield{part}}
    {}%
  \newunit
  \printfield{volumes}%
  \newunit\newblock
  \usebibmacro{series+number}%
  \newunit\newblock
  \printfield{note}%
  \newunit\newblock
  \usebibmacro{publisher+location+date}%
  \newunit\newblock
  \usebibmacro{chapter+pages}%
  \newunit
  \printfield{pagetotal}%
  \newunit\newblock
  \usebibmacro{doi+eprint+url}%
  \newunit\newblock
  \usebibmacro{addendum+pubstate}%
  \setunit{\bibpagerefpunct}\newblock
  \usebibmacro{pageref}%
  \setunit{\addcomma\space}
  \usebibmacro{date}
  \usebibmacro{finentry}}
%  
%
 \DeclareBibliographyDriver{online}{%
   \usebibmacro{bibindex}%
   \usebibmacro{begentry}%
   \ifnameundef{author}
    {\printtext{Autor unbekannt}}
    {
		\usebibmacro{author/editor+others/translator+others}%    
    }%
   \setunit{\labelnamepunct}\newblock
   \usebibmacro{title}%
   \newunitpunct
   \usebibmacro{url+urldate}%
   %\usebibmacro{addendum+pubstate}%
   \usebibmacro{finentry}}  
%%%%%%%%%%%%%%%%%%%%%%%%%%%%%%%%%%%%%%%%%%%%%%%%%%%%%%%%%%%






%%%%%%%%%%%%%%%%%%%%%%%%%%%%%%%%%%%%%%%%%%%%%%%%%%%%%%%%%%%
% Eigene Befehle 
% Matrix
\newcommand{\mat}[1]{
      {\textbf{#1}}
}
\newcommand{\todobox}[1]{
      {\colorbox{red}{ TODO: #1 }}
}
\newcommand{\todo}[1]{
      {\color{red}{ TODO: #1}} \normalfont
}
\newcommand{\info}[1]{
      ({\color{blue}{INFO: #1}}\normalfont)
}
\newcommand{\code}[1]{
      {\ttfamily{#1}}
}


\usepackage{suffix}		%newcommand with suffix (*)
% Referenzierung auf subfigures 
\newcommand\sref[1]{\protect\subref{#1}}
\WithSuffix\newcommand\sref*[1]{\protect\subref*{#1}}







%%%%%%%%%%%%%%%%%%%%%%%%%%%%%%%%%%%%%%%%%%%%%%%%%%%%%%
% Deutsche Begriffe fuer Pseudocode (algorithm)
%
% \floatname{algorithm}{Algorithmus}
% \renewcommand{\algorithmicrepeat}{\textbf{wiederhole}} 
% \renewcommand{\algorithmicuntil}{\textbf{bis}} 
% \renewcommand{\algorithmicfor}{\textbf{für}} 
% \renewcommand{\algorithmicend}{\textbf{}} 
% \renewcommand{\algorithmicdo}{} 

% \algnewcommand\algorithmicforeach{\textbf{für jedes}}
% \algdef{S}[FOR]{ForEach}[1]{\algorithmicforeach\ #1\ \algorithmicdo}






%%%%%%%%%%%%%%%%%%%%%%%%%%%%%%%%%%%%%%%%%%%%%%%%%%%%%%
% Groessenanpassungen
%
\setlength{\unitlength}{1cm}
\setlength{\oddsidemargin}{0.3cm}
\setlength{\evensidemargin}{0.3cm}
\setlength{\textwidth}{15.5cm}
\setlength{\topmargin}{-1.2cm}
\setlength{\textheight}{23cm}
\columnsep 0.5cm


%%%%%%%%%%%%%%%%%%%%%%%%%%%%%%%%%%%%%%%%%%%%%%%%%%%%%%
% Silbentrennung
%\hyphenation{Über-le-bens-wahr-schein-lich-keit}

\tikzset{
block/.style={
  rectangle,
  draw,
  text centered,
  rounded corners,
 },
line/.style={draw, -latex'},
edge/.style={draw}
}